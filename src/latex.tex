\documentclass{article}
\usepackage[T1]{fontenc}
\usepackage{amsmath}
\usepackage{amssymb}
\usepackage{textcomp}
\usepackage{graphicx} % Required for inserting images
\usepackage[a4paper, total={6in, 8in}]{geometry}
\title{Algorytmika ekonomiczna 3}
\author{Mikołaj Zakrzewski i Jakub Kołaczyński}
\date{marzec 2023}
\newcommand{\Le}{L_{\forall}}
\begin{document}

\maketitle
\section*{1.1}
    Opiszmy na początek taką funkcję krop po kroku dla osoby broniącej.\\
    1. Obliczamy na podstawie strategi mieszanej atakującego prawdopodobieństwo zaatakowania przez niego poszczególnych pól.\\
    2. Obliczamy oczekwianą stratę przy braku obrony danego pola i ataku atakującego, mnożąc prawdopodobieństwo zaatakowania poprzez wartość pola bitwy.\\
    3. Obrońca mając teraz k zasobów, będzie bronił k pól z największą oczekiwaną wartością straty, gdyby bronił innego pola, które nie należy do k pól z największą oczekiwaną wartością straty, to widzimy łatwo, że nie jest to najlepsza strategia czysta, gdyż może zmniejszyć swoją sumaryczną oczekwianą stratę poprzez bronienie bardziej wartościowego pola.
    \\
    Dla atakującego \\
    1. Obliczamy na podstawie strategi mieszanej broniącego prawdopodobieństwo obrony przez niego poszczególnych pól.\\
    2. Następnie by obliczyć wartość oczekiwaną danego pola mnożemy prawdobodobieństwo braku obrony danego pola przez wartość danego pola.
    3. Następnie atakujący mając k zasobów atakuje k pól z największą wartością oczekiwaną.\\
\section*{1.2}

\begin{flushleft}
Opiszemy tu jak działa ta funkcja dla obrońcy, gdyż dla atakującego jest ona zaimplementowana analogicznie.\\
    1. Mając oby dwie strategie liczymy ich wynik.\\
    2. Następnie generujemy strategię czystą dla obrońcy, na podstawie strategi mieszanej atakującego.\\
    3. Obliczamy wynik dla nowo otrzymanej strategi czystej obrońcy i strategi mieszanej atakującego.\\
    4. Od tak obliczonego wyniku odejmujemy wynik z podpunktu pierwszego. Tak otrzymana wartość jest naszym epsilonem.
\end{flushleft}

\section*{2.4}
\begin{flushleft}
    Dla atakującego szacowana wartość gry zawsze jest nieujemna, gdyż może on zdobywac tylko punkty, a nie tracic ich. Dla broniącego jest on nieujemny, gdy ma on dość surowców, by bronić wszystkie pola z niezerową wartością, gdyby miał on mniej surowców, to atakujący może atakować każde pole z takim samym prawdopodobieństwem i wartośc gry broniącego będzie ujemna.
\end{flushleft}

\end{document}
